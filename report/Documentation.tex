
%----------------------------------------------------------------------------------------
%	PACKAGES AND OTHER DOCUMENT CONFIGURATIONS
%----------------------------------------------------------------------------------------

\documentclass[paper=a4, fontsize=11pt]{scrartcl} % A4 paper and 11pt font size

\usepackage[T1]{fontenc} % Use 8-bit encoding that has 256 glyphs
\usepackage{fourier} % Use the Adobe Utopia font for the document - comment this line to return to the LaTeX default
\usepackage[english]{babel} % English language/hyphenation
\usepackage{amsmath,amsfonts,amsthm} % Math packages

\usepackage{lipsum} % Used for inserting dummy 'Lorem ipsum' text into the template

\usepackage[utf8]{inputenc}

\usepackage{sectsty} % Allows customizing section commands
\allsectionsfont{\centering \normalfont\scshape} % Make all sections centered, the default font and small caps

\usepackage{algorithm}
\usepackage{algorithmic}
\usepackage[linesnumbered,ruled]{algorithm2e}
\usepackage{multirow}
\usepackage{graphicx}

\newtheorem{theorem}{Theorem}[section]
\newtheorem{corollary}{Corollary}[theorem]
\newtheorem{lemma}[theorem]{Lemma}

\usepackage{fancyhdr} % Custom headers and footers
\pagestyle{fancyplain} % Makes all pages in the document conform to the custom headers and footers
\fancyhead{} % No page header - if you want one, create it in the same way as the footers below
\fancyfoot[L]{} % Empty left footer
\fancyfoot[C]{} % Empty center footer
\fancyfoot[R]{\thepage} % Page numbering for right footer
\renewcommand{\headrulewidth}{0pt} % Remove header underlines
\renewcommand{\footrulewidth}{0pt} % Remove footer underlines
\setlength{\headheight}{13.6pt} % Customize the height of the header

\numberwithin{equation}{section} % Number equations within sections (i.e. 1.1, 1.2, 2.1, 2.2 instead of 1, 2, 3, 4)
\numberwithin{figure}{section} % Number figures within sections (i.e. 1.1, 1.2, 2.1, 2.2 instead of 1, 2, 3, 4)
\numberwithin{table}{section} % Number tables within sections (i.e. 1.1, 1.2, 2.1, 2.2 instead of 1, 2, 3, 4)

\setlength\parindent{0pt} % Removes all indentation from paragraphs - comment this line for an assignment with lots of text


%----------------------------------------------------------------------------------------
%	TITLE SECTION
%----------------------------------------------------------------------------------------

\newcommand{\horrule}[1]{\rule{\linewidth}{#1}} % Create horizontal rule command with 1 argument of height

\title{	
\normalfont \normalsize 
\textsc{University of Zagreb, Faculty of Electronics and Computing} \\ [25pt] % Your university, school and/or department name(s)
\horrule{0.5pt} \\[0.4cm] % Thin top horizontal rule
\huge Computing optimal tresholds for q-gram filters \\ % The assignment title
\horrule{2pt} \\[0.5cm] % Thick bottom horizontal rule
}

\author{Stipe Kuman, Dino Radaković, Leon Rotim} % Your name

\date{\normalsize\today} % Today's date or a custom date

\begin{document}

\maketitle % Print the title

%----------------------------------------------------------------------------------------
% PROBLEM 1
%----------------------------------------------------------------------------------------

\section{Introduction into the problem}

The focus of this assignment was finding optimal thresholds for $q$-gram filters using the algorithm described in \cite{njihovPaper}. The emphasis is on gapped $q$-grams, discontinuous patterns of $q$ matching positions which can be used to efficiently discard large portions of text prior to using of actual string search algorithms to find matches (filtering), using a metric called $q$-gram similarity (the number of $q$-grams shared by two strings). Gapped $q$-grams were shown to be more efficient than contiguous $q$-grams in the right conditions, according to \cite{njihovPaper}. However, computing an optimal threshold for gapped $q$-gram filters is also more difficult, as opposed to a closed-formula in the case of contiguous $q$-grams. 
This project was implemented in C++ (adhering to the C++$11$ standard) and can be built on the Bio-Linux \cite{biolinux} platform, without any additional dependencies.
In the next section we describe the problem, after which we'll describe the original algorithm, alongside our implementation details, followed by the results
produced using our implementation.

%------------------------------------------------

\subsection{Problem description}

First off, we will describe the notation used in this assignment. \textit{T} is the text file, \textit{P} is the given pattern and \textit{S} stands 
for all the substrings in \textit{T} that match the given pattern \textit{P} within a Hamming distance of \textit{k} (number of non-matching characters). 
The length of the $q$-gram filter is $q$ and the threshold is represented by \textit{t}. Finally, \textit{m} is the length of \textit{P} and \textit{S}. \\
The goal of the $q$-gram filter is to reduce the number of potential matches that need to be compared to the given pattern \textit{P}. 
The threshold variable \textit{t} defines the minimum number of $q$-gram matches between the pattern \textit{P} and 
a substring so that the substring would be considered as a potential match. A low value of \textit{t} will result in too many potential matches, 
while a too high value may overlook some potential matches which would make the filter lossy. Only filters that label all the matches as 
potential matches will be considered in the assignment.\\
As stated previously, this assignment will focus on gapped $q$-grams. In gapped $q$-gram filters the problem is much more complex
because a closed form formula has not been found so we have to use an algorithm to find the optimal treshold.\\
Because we are dealing with gapped $q$-grams, which we represent with sets, we will have to define a $q$-shape and
 some additional terms. The size of a set Q is the number of indexes that it has, while the span is the distance between te lowest
and highest member of the set, or in other words \textit{span(Q) = maxQ - minQ + 1} . The \textit{(q,s)-shape} of a q-gram is a set Q with it's
size \textit{q} and span \textit{s}. For example, if we have a shape Q=\{0,1,4,6,7\} it's size is 5 and it's span is 8 so it is a (5,8)-shape.


%------------------------------------------------

\section{Computing the optimal threshold}
\label{sec:compopt}
\subsection{The recurrence}
  \label{subsec:rec}
  A gapped $q$-gram can be represented by a set of indices denoting matching positions, $Q$, over a \textit{span} $s = \max_{i \in Q} i - \min_{i \in Q} i$.
With this representation in mind, we can define the optimal threshold $t_Q(m, k)$ for a given gapped $q$-gram $Q$, on any pattern and its potential match of length $m$, where the the Hamming distance between the two equals $k$.

In order to devise an algorithm in terms of optimal substructure, the authors \cite{njihovPaper} also introduce a third parameter $M \subseteq \left\{1, 2, \dots, s - 1\right\}$, which represents matches between the two strings at the last $s - 1$ positions. As an example, $ M = \left\{1, 3, 4 \right\} $ with $ s = 6 $ would represent a match between the strings' characters at positions $(m - 4)$,  $(m - 2)$ and $(m - 1)$. \\

Let $ cond(b) $ be $ 1 $ if the boolean expression $b$ holds and $ 0 $ otherwise. Let $M \oplus x = \left\{ m + x | m \in M \right\}$ ($M \ominus x$ being the inverse). S The introduced conditional threshold can be computed via the following recurrence: \\
\begin{equation*}
  \begin{array}{ll@{}ll}
    t_Q(s - 1, j, M) & =  0 \\
    t_Q(i, j, M) & = \min 
      \begin{cases} 
        t_Q(i - 1, j - cond(s - 1 \not \in M), (M \cup \left\{0\right\} \setminus \left\{s - 1\right\}) \oplus 1) \\ \hspace{30pt} + cond(Q \subseteq M \cup \left\{0\right\}) \\ 
        t_Q(i - 1, j - cond(s - 1 \not \in M), (M \setminus \left\{s - 1\right\}) \oplus 1)
      \end{cases}
  \end{array}
\end{equation*}
\\

The recurrence above is correct if the following three conditions hold:
\begin{enumerate} \label{rec:correctness}
  \item $ s \leq i \leq m $
  \item $ 0 \leq j \leq k$
  \item $|M| \geq s - 1 - j$
\end{enumerate}
 The first of the given correctness conditions needs to be satisfied because a $q$-gram of span $s$ wouldn't fit in the given string unless its length were at least as large as the $q$-gram's span. The third condition must hold due to the lowest possible number of matches at the last $ s - 1 $ positions with $ j $ mismatches in the whole string being $ s - 1 - j $, thus representing the case in which all of the mismatches occur in the last $ s - 1 $ characters of the string. \\

Another property of the recurrence that's worth mentioning is the case where $j = 0$. Obviously, for two strings to match completely, the $q$-gram similarity must be maximal. This also implies $M = {1, 2, \dots,  s - 1}$, and, implicitly, avoiding the second case of the recursion, due to the fact that it generates an invalid set of matches (which violates the third of he given correctness conditions). Therefore, the recurrence will grow in value with $ j = 0 $, dropping towards the final solution, for given $ m $, $ k $, and $ M$. The recurrence itself can be thought of as a sliding window, going from the end of the original string of length $m$, towards its beginning, enumerating all possible combinations of mismatches in the process (hence the binary choice at each step).  \\

Once computed, the conditional thresholds can be used to derive the optimal threshold for $ m $ and $ k $:
\begin{equation*}
  t_Q(m, k) = \min_{M \subseteq \left\{1, 2, \dots, s - 1\right\}, |M| \geq s - 1 - k} t_Q(m, k, M) 
\end{equation*} \\
The authors claim the impossibility of computing $t_Q(m, k)$ from $t_Q(m - 1, k$) alone \cite{njihovPaper}. \\

Judging by the branching required to compute each $t_Q(i, j, M)$ recursively, the given recurrence would obviously require asymptotically exponential time to compute.
However, it does exhibit an \textit{optimal substructure} -- each step in the recurrence (the $\min$-aggregation) takes $\mathcal{O}(1)$ time. The subproblems are also \textit{overlapping}, meaning that a same set of recurrence parameters ought to appear multiple times if the recurrence is solved recursively, in a naïve manner.

\subsection{A dynamic programming approach}
\label{subsec:dynprog}
Exploiting the optimal substructure and overlapping subproblems of the recurrence given in subsection \ref{subsec:rec} leads to a dynamic programming algorithm.
Simply put, the algorithm is based on initializing a three-dimensional array with the base case, setting positions $\left[s - 1, j, M \right]$ to $0$ and applying the recurrence in a bottom-up fashion, starting from positions $ [s, *, *] $. We represent each set of matches $M$ using a $64$-bit integer, due to the maximum feasible (in terms of complexity and computational resources) spans, as stated in the original paper's experiments \cite{njihovPaper}, do not exceed $64$. Such a representation allows us to perform set arithmetic efficiently, using bitwise operations, which, when compiled, tend to map to single machine instructions. As an example, `$\oplus$` amounts to a single bit-shifting operation. \\

When encountering a problem (a set of parameters $i$, $j$ and $M$) which doesn't satisfy the one of the correctness criteria listed in \ref{rec:correctness}, we mark it as incorrect. When one of the subproblems of a problem happens to be marked as incorrect, we ignore it and assign the value based on the other subproblem. When both subpro<blems of a (otherwise correct) problem happen to be marked as incorrect, we mark the problem itself as incorrect, too. \\

\subsection{Asymptotic complexity}
\label{subsec:complexity}
The dynamic programming approach significantly reduces asymptotic time complexity by eliminating $\mathcal{O}(2^m)$ as a factor, reducing it to $\mathcal{O}(m)$.
Solving a problem based on two subproblems is done in $\mathcal{O}(1)$ with a single comparison of the subproblems' solutions ($\min$). 
In order to compute $t_Q(m, k)$, one needs to compute $t_Q(m, k, M)$, for all valid values of $M$. \\

For given $m$ and $k$ on a $q$-gram of span $s$, there are $\sum_{e = 0}^k {s - 1 \choose e}$ possible arrangements of $k$ mismatches on $(s - 1)$ positions (complementary to sets $M$), which is $\mathcal{O}(2^k)$ \footnote{the authors provide tighter bounds for both time and space complexity using the sum of binomial coefficients \cite{njihovPaper}}. This ultimately leads to asymptotic time complexity of $\mathcal{O}(mk^2 2^k)$, due to our set operations on $M$ (except copying) being $\mathcal{O}(1)$, whereas copying is done in $\mathcal{O}(k)$ time, which accounts for the additional factor $\mathcal{O}(k)$. Different types of sets (possibly purely functional reference-based sets) would account for different, perhaps improved asymptotic time complexities. 

We avoid memorizing the whole three-dimensional dynamic programming array, and only keep each $i$-row and $(i - 1)$-row instead, because we ultimately only need to compute the row $t_Q(i, *, *)$. This leads to space complexity of $\mathcal{O}(k^22^k)$. The additional $\mathcal{O}(k)$ factor comes from sets used to represent matches ($M$).  



\subsection{Pruning method}
\label{subsec:pruning}

Even with this reasonably fast dynamic programming algorithm, computing all
possible gapped Q-grams with positive threshold for parameters $m=50,\
k=\{4,5\}$ is still a challenging task. Using brute force method, calculating
threshold for all shapes that have $m=50$ and $k=\{4,5\}$ and leaving only
those for which positive threshold is computed, would leave us with a search
space of $2^{50} (\approx 10^{15}$) thresholds to compute. Since that kind of search
space is clearly unfeasible we use following lemma as stated
in~\cite{njihovPaper} to reduce the search space.

\begin{lemma}
    \label{subSetLemma}
    If $Q' \subseteq Q$ then $t_{Q'}(m,k) \geq t_{Q}(m,k)$
\end{lemma}

Using lemma~\ref{subSetLemma} we propose a simple pruning technique which we
used to speed up computing all Q-shapes with positive thresholds.

We introduce parameter $MS$ as the maximal span size of all Q-grams that will
considered. Obviously $MS \leq m$. This parameter is important because its value
drastically changes search space.

Let $Set_{current}$ be a set which contains all Q-grams with size $q$, which
respective thresholds had been computed.
Next we define two empty sets, $Set_{next}$ and $Set_{forbidden}$, which will
contain Q-grams of size $q+1$. After running pseudocode~\ref{alg:pruning}
$Set_{next}$ will contain all Q-grams of size $q+1$ that can not be eliminated
with lemma~\ref{subSetLemma}. 

\begin{algorithm}[H]
\caption{Generating candidate set of Q-grams which may have positive threshold}
\label{alg:pruning}
\underline{function generateNextCancidateSet} $(Set_{current}, MS)$\;
\SetKwInOut{Input}{Input}
\SetKwInOut{Output}{Output}

\Input{$Set_{current}$ - a set of Q-grams of size q, MS - maximal value of span
for all considered Q-grams}

\Output{$Set_{next}$ - set of Q-grams of size q+1 that can't be eliminated using
lemma~\ref{subSetLemma}}
\begin{algorithmic}
	 \STATE $Set_{next} \gets \emptyset$
	 \STATE $Set_{forbidden} \gets \emptyset$
     
     \FOR{each Q-gram $Q_{i} \in Set_{current}$}
         \FOR{ $i \in \{2,\dots,MS\} \setminus Q_{i}$ }
				 \STATE { $Q_{j}' = Q_{i} \cup i$
             	 	\IF{ threshold($Q_{i}$) ==  0 }
					\STATE { $Set_{next} \gets Set_{next} \setminus Q_{j}'$ }
	                \STATE { $Set_{forbidden} \gets Set_{forbidden} \cup Q_{j}'$ }
                    \ELSIF { $Q_{j}' \notin Set_{forbidden}$ }
                    \STATE { $Set_{next} \gets Set_{next} \cup Q_{j}'$ }
                    \ENDIF
                    }
         \ENDFOR
     \ENDFOR
\end{algorithmic}
return $Set_{next}$
\end{algorithm}

In more detail if some $Q_{i}$ has a non positive threshold none of his $Q_{j}'$
supersets may appear in $Set_{next}$. We ensure that by explicitly removing
all such $Q_{j}'$ from $Set_{next}$ and adding it to $Set_{forbidden}$.
$Set_{forbidden}$ is needed because there can be multiple $Q_{i}$ with same
superset $Q_{j}'$ and if there is at least one such $Q_{i}$ with
$t_{Q_{i}}$ $= 0$ we mustn't include $Q_{j}'$ in $Set_{next}$.

This rule follows from lemma~\ref{subSetLemma} from which we know that
$t_{Q_{j}'}$ must be $0$. $Set_{forbidden}$ therefore ensures that some
Q-gram $Q_{j}'$ will not be included in $Set_{next}$ if we earlier determined
that its threshold must be $0$.

Using this method (Algorithm~\ref{alg:pruning}) we can compute thresholds for
all Q-grams with positive thresholds with $m=50$ and $k=\{4,5\}$ as presented in following
algorithm~\ref{alg:pruningLoop}.

\begin{algorithm}[H]
\caption{Pruning technique for computing all Q-grams with positive threshold}
\label{alg:pruningLoop}

\underline{function findAllQgramsWithPositiveThreshold } $(m,k,MS)$\;

\SetKwInOut{Input}{Input}

\Input{$m$ - size of a pattern, $k$ - number of miss-matches, $MS$ - maximal
value of span for all considered Q-grams}

\begin{algorithmic}
	\STATE $Set_{current} \gets \{Q_{start}\}$, where $Q_{start} = \{0\}$
	\WHILE{$Set_{current} \neq \emptyset}
		\FOR{each $Q_{i} \in Set_{current}$}
			\IF {threshold($Q_{i}$) > 0}
			\STATE add pair ($Q_{i}$, threshold($Q_{i}$)) to results
			\ENDIF
			\STATE $Set_{current} \gets$ generateNextCancidateSet($Set_{current}$, $MS$)
		\ENDFOR	
\end{algorithmic}

\end{algorithm}

Using the method above we are basically performing space search where in each
iteration $Set_{current}$ contains $N$ Q-gram shapes whose threshold values are
to be evaluated and, if positive, submitted to results.

With this simple pruning method we drastically reduced search space by
several orders of magnitude. Leaving about $10^{8}$ thresholds to compute in order to
acquire all Q-shapes with positive thresholds with $m=50$ and $k=\{4,5\}$.

%----------------------------------------------------------------------------------------

\subsection{Results and performance}

Here we present performance and results obtained using dynamic programming
algorithm and pruning method presented in sections~\ref{subsec:dynprog}
and~\ref{subsec:pruning} respectively.

Amount of data that is needed while running dynamic programming recurrence
depends on different values of $k$ and $span$. While computing all positive shapes for
$m=50$ and $k=\{4,5\}$ we ran cases $k=4$ and $k=5$ separately. So when running
computation value of $k$ will be fixed and there will be at most $50$ different
values of $span$. Before we start computing threshold and running our pruning
method we precompute data for all different values of $span$ that will be used
in dynamic programming recurrence resulting in drastic time performance gain.

Precomputation finishes within $30$ seconds (for all values of span). It would
be really expensive recomputing this data for each shape when computing its threshold.
To show significance of precomputation we generated $1000$ random Q-gram
examples with span $50$. Using naive implementation, without precomputation, it
took $3421.21$ seconds to compute all $1000$ thresholds. Using precomputation
all thresholds were computed in $13.78$ seconds.

As mentioned in section~\ref{subsec:dynprog} memory usage depends on parameters
$m$ an $k$. For $k=5$ precomputed data for DP recurrence mentioned
earlier sums up to $\approx 1GB$ of RAM which is more then feasible for most PC
we have in use today. With $k=4$ memory usage is considerably smaller using only
$\approx 120MB$ of RAM.

To get a better view of time performance we generated random Q-grams
shapes and measured time of their respective threshold computation. For each
value of $span$ ranging from $span = 10$ up to $span=50$ we generated $100$
random shapes. Time measurements are presented in Table~\ref{tab:measurmentTime}
and graphically on Figure~\ref{fig:timeGraph}.

\begin{table}[H]
\centering
\begin{tabular}{|c|c|c|c|c|}
	\hline
	\multirow{ 2}{*}{Span} & \multicolumn{4}{|c|}{CPUs} \\ 
	 & 1 & 2 & 3 & 4 \\ \hline
	10 & 136627.05 & 126988.41 & 90977.85 & 68577.47 \\ \hline
11 & 196651.00 & 183161.20 & 128614.36 & 100547.80 \\ \hline
12 & 285034.06 & 262230.65 & 183353.10 & 143803.90 \\ \hline
13 & 533504.01 & 368306.46 & 254942.96 & 201404.48 \\ \hline
14 & 535976.65 & 499980.73 & 353007.54 & 276948.10 \\ \hline
15 & 699336.82 & 681061.92 & 480749.87 & 372993.07 \\ \hline
16 & 947119.22 & 910288.81 & 639657.39 & 520388.72 \\ \hline
17 & 1237340.15 & 1203045.88 & 852631.28 & 644182.21 \\ \hline
18 & 1611819.83 & 1532784.16 & 1079991.32 & 829194.88 \\ \hline
19 & 2063239.34 & 2064853.58 & 1380863.72 & 1049174.62 \\ \hline
20 & 2407162.88 & 2491824.53 & 1839472.54 & 1307171.31 \\ \hline
21 & 3260196.25 & 2987138.81 & 2064527.85 & 1606242.88 \\ \hline
22 & 3497564.29 & 3593302.42 & 2241717.10 & 1959988.34 \\ \hline
23 & 4324972.57 & 4422506.20 & 3122416.46 & 2377207.61 \\ \hline
24 & 5068956.12 & 5348155.24 & 3645893.90 & 2896774.42 \\ \hline
25 & 5992657.58 & 6217346.25 & 4414702.87 & 3496285.09 \\ \hline
26 & 8976574.14 & 7492514.17 & 5405795.41 & 4268631.13 \\ \hline
27 & 8873869.08 & 8699772.02 & 6426091.94 & 5023584.94 \\ \hline
28 & 10813056.63 & 10435265.04 & 7868119.10 & 6403527.86 \\ \hline
29 & 13002752.35 & 12339981.24 & 9297495.81 & 7423738.08 \\ \hline
30 & 15665078.52 & 14137244.21 & 11287121.10 & 8520270.27 \\ \hline
31 & 17312777.16 & 12356572.02 & 12892359.47 & 11897454.56 \\ \hline
32 & 19742031.53 & 17242779.55 & 16276930.99 & 13006160.85 \\ \hline
33 & 21898560.03 & 20751728.32 & 18210394.64 & 13350733.70 \\ \hline
34 & 25402239.90 & 21747892.16 & 18888410.65 & 15011996.75 \\ \hline
35 & 26162620.93 & 21964907.21 & 20115109.16 & 19169146.94 \\ \hline
36 & 28575425.98 & 25511574.68 & 24705952.64 & 20285299.67 \\ \hline
37 & 32071472.73 & 29122202.36 & 25281911.88 & 23319477.19 \\ \hline
38 & 33983451.96 & 30944120.06 & 28064518.11 & 24578031.54 \\ \hline
39 & 35790107.91 & 29812912.16 & 29916421.11 & 31401271.26 \\ \hline
40 & 36941464.55 & 33435276.40 & 26194934.28 & 28235519.12 \\ \hline
41 & 38408097.05 & 31553883.27 & 31017630.10 & 28979277.53 \\ \hline
42 & 41310449.19 & 32248354.62 & 32557746.20 & 26518729.54 \\ \hline
43 & 41856719.41 & 36764295.47 & 33454317.76 & 32855935.33 \\ \hline
44 & 43877883.72 & 35791089.10 & 30272635.14 & 31065587.07 \\ \hline
45 & 40160153.92 & 31261790.44 & 31136853.20 & 30336145.68 \\ \hline
46 & 38423015.43 & 33199604.60 & 28637212.85 & 22360396.87 \\ \hline
47 & 34861463.42 & 29926723.66 & 27191496.25 & 25294109.14 \\ \hline
48 & 28567696.51 & 23252670.42 & 22592821.10 & 20204953.65 \\ \hline
49 & 22246720.09 & 19196621.75 & 15977201.05 & 15551445.28 \\ \hline

\end{tabular}
\caption{Time needed to compute a single shape with given span (averaged on
$100$ evaluation) depending on number of used threads in nanoseconds}
\label{tab:measurmentTime}
\end{table}

\begin{figure}[H]
\centering
\includegraphics[width=.6\linewidth]{time.png}
\label{fig:timeGraph}
\caption{Time needed to compute a single shape with given span (averaged on
$100$ evaluations) depending on number of used threads (CPUs) in nanoseconds}
\end{figure}

While the implemented dynamic programming algorithm is reasonably fast,
the number of thresholds that has to be computed is extremely large (up to
$10^8$). Therefore we decided to achieve linear factor speed-up using multi-threading. Time
performances acquired are presented~\ref{tab:measurmentTime}.

To reduce search space we first ran computation considering only shapes with
$span < 30$. With parameters $m=50$, $k=5$ and $MS=30$ (maximal span of
considered shapes) a total of $1588439$ shapes with positive threshold were
evaluated. Experiments were ran on Intel $i7$ using $4$ threads. Complete search
time was $15906$ seconds ($4.4183$ hours).

With parameters $m=50$, $k=4$ and $MS=30$ a total of $13104841$. Complete search
time was $19173.528$ seconds ($5.3259799$ hours). 

Results obtained using presented methods are showed in tables~\ref{tab:res1}
and~\ref{tab:res2}. We note that tables do not contain threshold values for all
possible shapes. Even with our effective pruning technique number of thresholds
that has to be computed is still large. Due to the imposed deadline and
shortage of computational power experiments that were ran with parameters
$m=50$, $k=\{4,5\}$ and $MS = 50$ didn't complete the entire search space. For
$k=4$ only Q-gram shapes with $q \leq 6$ were evaluated and for $k=5$ only
Q-gram shapes with $q \leq 6$ were evalated. Results for complete search space
can surely be computed in matter of days.

\begin{table}[H]
\centering
\begin{tabular} {|c|c|c|c|c|c|c|c|c|c|c|c|c|c|}
	\hline
	\multirow{ 2}{*}{Span} & \multicolumn{13}{|c|}{q} \\ 
	  & 2 &	3 & 4 &	5 & 6 & 7 & 8 & 9 &	10 & 11 & 12 & 13 & 14 \\ \hline
	2 & 41 & - & - & - & - & - & - & - & - & - & - & - & - \\
	3 & 40 & 36 & - & - & - & - & - & - & - & - & - & - & - \\
	4 & 39 & 35 & 31 & - & - & - & - & - & - & - & - & - & - \\
	5 & 38 & 34 & 30 & 26 & - & - & - & - & - & - & - & - & - \\
	6 & 37 & 33 & 29 & 25 & 21 & - & - & - & - & - & - & - & - \\
	7 & 36 & 32 & 28 & 24 & 20 & 16 & - & - & - & - & - & - & - \\
	8 & 35 & 31 & 27 & 23 & 19 & 15 & 11 & - & - & - & - & - & - \\
	9 & 34 & 30 & 26 & 22 & 18 & 14 & 10 & 6 & - & - & - & - & - \\
	10 & 33 & 29 & 25 & 21 & 17 & 13 & 9 & 5 & 1 & - & - & - & - \\
	11 & 32 & 28 & 24 & 20 & 17 & 14 & 10 & 7 & 4 & 0 & - & - & - \\
	12 & 31 & 27 & 23 & 20 & 17 & 13 & 10 & 8 & 5 & 2 & 0 & - & - \\
	13 & 30 & 26 & 22 & 19 & 16 & 13 & 10 & 8 & 6 & 3 & 1 & 0 & - \\
	14 & 29 & 25 & 21 & 18 & 15 & 12 & 10 & 8 & 5 & 4 & 2 & 1 & 0 \\
	15 & 28 & 24 & 20 & 17 & 14 & 12 & 9 & 7 & 5 & 3 & 2 & 1 & 0 \\
	16 & 27 & 23 & 19 & 16 & 14 & 11 & 9 & 7 & 5 & 3 & 2 & 1 & 0 \\
	17 & 26 & 22 & 18 & 16 & 13 & 11 & 9 & 7 & 5 & 4 & 2 & 1 & 0 \\
	18 & 25 & 21 & 17 & 15 & 12 & 10 & 8 & 6 & 5 & 3 & 2 & 1 & 0 \\
	19 & 24 & 20 & 17 & 14 & 12 & 9 & 7 & 6 & 4 & 3 & 2 & 1 & 1 \\
	20 & 23 & 19 & 16 & 13 & 11 & 9 & 7 & 5 & 4 & 3 & 2 & 1 & 0 \\
	21 & 22 & 18 & 15 & 12 & 10 & 8 & 6 & 5 & 3 & 2 & 2 & 1 & 0 \\
	22 & 21 & 17 & 15 & 12 & 9 & 7 & 6 & 4 & 3 & 2 & 1 & 1 & 0 \\
	23 & 20 & 17 & 14 & 12 & 9 & 7 & 5 & 4 & 2 & 2 & 1 & 0 & 0 \\
	24 & 19 & 17 & 15 & 12 & 10 & 7 & 5 & 4 & 2 & 1 & 1 & 0 & 0 \\
	25 & 20 & 17 & 15 & 12 & 10 & 7 & 5 & 4 & 3 & 2 & 1 & 0 & 0 \\
	26 & 21 & 17 & 16 & 12 & 11 & 8 & 6 & 4 & 3 & 2 & 1 & 1 & 0 \\
	27 & 20 & 16 & 16 & 12 & 12 & 8 & 7 & 5 & 4 & 3 & 2 & 1 & 0 \\
	28 & 19 & 15 & 15 & 11 & 11 & 8 & 7 & 6 & 4 & 3 & 2 & 1 & 1 \\
	29 & 18 & 14 & 14 & 10 & 10 & 8 & 7 & 5 & 4 & 3 & 2 & 1 & 1 \\
	30 & 17 & 13 & 13 & 9 & 9 & 7 & 6 & 4 & 4 & 3 & 2 & 1 & 1 \\
	31 & 16 & 12 & 12 & 9 & 9 & ? & ? & ? & ? & ? & ? & ? & ? \\
	32 & 15 & 11 & 11 & 9 & 7 & ? & ? & ? & ? & ? & ? & ? & ? \\
	33 & 14 & 12 & 10 & 8 & 8 & ? & ? & ? & ? & ? & ? & ? & ? \\
	34 & 13 & 12 & 9 & 8 & 8 & ? & ? & ? & ? & ? & ? & ? & ? \\
	35 & 12 & 12 & 8 & 8 & 8 & ? & ? & ? & ? & ? & ? & ? & ? \\
	36 & 11 & 11 & 7 & 7 & 7 & ? & ? & ? & ? & ? & ? & ? & ? \\
	37 & 10 & 10 & 8 & 6 & 6 & ? & ? & ? & ? & ? & ? & ? & ? \\
	38 & 9 & 9 & 8 & 5 & 5 & ? & ? & ? & ? & ? & ? & ? & ? \\
	39 & 8 & 8 & 8 & 5 & 4 & ? & ? & ? & ? & ? & ? & ? & ? \\
	40 & 7 & 7 & 7 & 5 & 3 & ? & ? & ? & ? & ? & ? & ? & ? \\
	41 & 6 & 6 & 6 & 6 & 4 & ? & ? & ? & ? & ? & ? & ? & ? \\
	42 & 5 & 5 & 5 & 5 & 4 & ? & ? & ? & ? & ? & ? & ? & ? \\
	43 & 4 & 4 & 4 & 4 & 4 & ? & ? & ? & ? & ? & ? & ? & ? \\
	44 & 3 & 3 & 3 & 3 & 3 & ? & ? & ? & ? & ? & ? & ? & ? \\
	45 & 2 & 2 & 2 & 2 & 2 & ? & ? & ? & ? & ? & ? & ? & ? \\
	46 & 1 & 1 & 1 & 1 & 1 & ? & ? & ? & ? & ? & ? & ? & ? \\ \hline
\end{tabular}
\caption{Table showing maximal thresholds for Q-gram shapes with regard to
size($q$) and span of the shape for $m=50$ and $k=4$}
\label{tab:res1}
\end{table}

\begin{table}[H]
\centering
\begin{tabular} {|c|c|c|c|c|c|c|c|c|c|c|c|}
	\hline
	\multirow{ 2}{*}{Span} & \multicolumn{11}{|c|}{q} \\ 
	  & 2 &	3 & 4 &	5 & 6 & 7 & 8 & 9 &	10 & 11 & 12 \\ \hline
	2 & 39 & - & - & - & - & - & - & - & - & - & - \\
	3 & 38 & 33 & - & - & - & - & - & - & - & - & - \\
	4 & 37 & 32 & 27 & - & - & - & - & - & - & - & - \\
	5 & 36 & 31 & 26 & 21 & - & - & - & - & - & - & - \\
	6 & 35 & 30 & 25 & 20 & 15 & - & - & - & - & - & - \\
	7 & 34 & 29 & 24 & 19 & 14 & 9 & - & - & - & - & - \\
	8 & 33 & 28 & 23 & 18 & 13 & 8 & 3 & - & - & - & - \\
	9 & 32 & 27 & 22 & 18 & 14 & 9 & 5 & 0 & - & - & - \\
	10 & 31 & 26 & 21 & 18 & 13 & 10 & 6 & 3 & 0 & - & - \\
	11 & 30 & 25 & 20 & 16 & 13 & 10 & 7 & 4 & 2 & 0 & - \\
	12 & 29 & 24 & 19 & 16 & 12 & 9 & 7 & 4 & 2 & 0 & 0 \\
	13 & 28 & 23 & 19 & 15 & 12 & 9 & 6 & 4 & 2 & 1 & 0 \\
	14 & 27 & 22 & 17 & 14 & 11 & 8 & 6 & 4 & 2 & 1 & 0 \\
	15 & 26 & 21 & 17 & 13 & 10 & 8 & 5 & 3 & 2 & 1 & 0 \\
	16 & 25 & 20 & 16 & 13 & 10 & 7 & 5 & 3 & 2 & 1 & 0 \\
	17 & 24 & 19 & 15 & 12 & 9 & 7 & 5 & 3 & 2 & 1 & 0 \\
	18 & 23 & 18 & 14 & 11 & 8 & 6 & 4 & 3 & 2 & 1 & 0 \\
	19 & 22 & 17 & 14 & 11 & 8 & 6 & 4 & 2 & 1 & 1 & 1 \\
	20 & 21 & 16 & 13 & 10 & 7 & 5 & 3 & 2 & 1 & 1 & 0 \\
	21 & 20 & 15 & 12 & 9 & 7 & 5 & 3 & 2 & 1 & 0 & 0 \\
	22 & 19 & 15 & 12 & 9 & 6 & 4 & 2 & 1 & 1 & 0 & 0 \\
	23 & 18 & 15 & 12 & 9 & 6 & 4 & 2 & 1 & 0 & 0 & 0 \\
	24 & 18 & 15 & 13 & 9 & 7 & 4 & 2 & 1 & 0 & 0 & 0 \\
	25 & 19 & 15 & 13 & 9 & 7 & 4 & 3 & 1 & 1 & 0 & 0 \\
	26 & 20 & 15 & 14 & 9 & 8 & 5 & 3 & 2 & 1 & 0 & 0 \\
	27 & 19 & 14 & 14 & 9 & 9 & 6 & 4 & 2 & 1 & 1 & 0 \\
	28 & 18 & 13 & 13 & 8 & 8 & 5 & 4 & 3 & 2 & 1 & 0 \\
	29 & 17 & 12 & 12 & 8 & 8 & 5 & 5 & 2 & 2 & 1 & 0 \\
	30 & 16 & 11 & 11 & 7 & 6 & 4 & 4 & 2 & 2 & 1 & 0 \\
	31 & 15 & 10 & 10 & 7 & 7 & ? & ? & ? & ? & ? & ? \\
	32 & 14 & 10 & 9 & 7 & 5 & ? & ? & ? & ? & ? & ? \\
	33 & 13 & 11 & 8 & 6 & 6 & ? & ? & ? & ? & ? & ? \\
	34 & 12 & 11 & 7 & 6 & 6 & ? & ? & ? & ? & ? & ? \\
	35 & 11 & 11 & 6 & 6 & 6 & ? & ? & ? & ? & ? & ? \\
	36 & 10 & 10 & 6 & 5 & 5 & ? & ? & ? & ? & ? & ? \\
	37 & 9 & 9 & 7 & 4 & 4 & ? & ? & ? & ? & ? & ? \\
	38 & 8 & 8 & 7 & 4 & 3 & ? & ? & ? & ? & ? & ? \\
	39 & 7 & 7 & 7 & 4 & 3 & ? & ? & ? & ? & ? & ? \\
	40 & 6 & 6 & 6 & 4 & 2 & ? & ? & ? & ? & ? & ? \\
	41 & 5 & 5 & 5 & 5 & 3 & ? & ? & ? & ? & ? & ? \\
	42 & 4 & 4 & 4 & 4 & 3 & ? & ? & ? & ? & ? & ? \\
	43 & 3 & 3 & 3 & 3 & 3 & ? & ? & ? & ? & ? & ? \\
	44 & 2 & 2 & 2 & 2 & 2 & ? & ? & ? & ? & ? & ? \\
	45 & 1 & 1 & 1 & 1 & 1 & ? & ? & ? & ? & ? & ? \\ \hline
\end{tabular}
\caption{Table showing maximal thresholds for Q-gram shapes with regard to
size($q$) and span of the shape for $m=50$ and $k=5$}
\label{tab:res2}
\end{table}

%----------------------------------------------------------------------------------------


\subsection{Conclusion}

In order to compute optimal threshold for gapped q-grams we used the 
dynamic programming algorithm described in \cite{njihovPaper}, as well as a pruning technique, as described in
in \ref{subsec:dynprog} and \ref{subsec:pruning}, respectively. Even though an
implementation of the dynamic programming algorithm is a both practical and
asymptotic improvement as opposed to the trivial recursive approach, it's still
exponential (in terms of time and space complexity) in the number of mismatches
between the two strings, which requires significant amount of computational
power to compute in reasonable time. Had we employed more machines, we would
have evaluated a proportionally larger number of $q$-grams, thus (hopefully)
finding a proportionally larger number of those with non-negative optimal thresholds as well. This is due to the fact that the solution can be (almost) trivially parallelized. The results we've managed to compile so far match exactly with those shown in the authors' original peer-revied paper, meaning our program can safely be used to find $q$-grams with non-negative thresholds which can be used to perform efficient filtering prior to string searching. We conclude with our conjecture that the greatest improvements in solving this problem can be done by either improving the asymptotic time complexity of the algorithm itself (with both a reasonable constant time overhead and space complexity) or implementing a solution which uses computers' caches in a more efficient way, in order to achieve a significant constant-time speedup.


\bibliographystyle{plain}
\bibliography{Documentation}

\end{document}